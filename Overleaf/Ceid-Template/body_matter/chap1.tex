\chapter{Εισαγωγή}

\section{Ορισμός του Προβλήματος}
Εφαρμογή τεχνικών βελτιστοποίησης σε πρόβλημα ταξινόμησης εικόνων κάνοντας χρήση
ενός Συνελικτικού Νευρωνικού Δικτύου

\section{Στόχοι της Διπλωματικής Εργασίας}

Γίνεται έρευνα σχετικά με τις διαφορετικές αρχιτεκτονικές νευρωνικών δικτύων και τις μεθόδους εκπαίδευσης, ωστόσο, μια άλλη κρίσιμη πτυχή των νευρωνικών δικτύων είναι, δεδομένου ενός εκπαιδευμένου δικτύου, η γρήγορη και ακριβής ταξινόμηση εικόνων. Σε αυτό το \en{project}, δίνετε ένα εκπαιδευμένο νευρωνικό δίκτυο που ταξινομεί εικόνες \(32\times32\) \en{RGB} σε 10 κατηγορίες. Οι εικόνες ανήκουν στο \en{dataset CIFAR-10 \cite{cifar-10}}. Μας δίνετε ο αλγόριθμος εμπρόσθιας διάδοσης του νευρωνικού δικτύου και τα τελικά βάρη του δικτύου.\cite{cs61c} Σκοπός είναι η βελτίωση της ταχύτητας της εμπρόσθιας διάδοσης ώστε να γίνει ταξινόμηση με πιο γρήγορο ρυθμό. 

Βασίστικε πάνω στο \en{project} του \en{Berkeley {\url{https://inst.eecs.berkeley.edu/~cs61c/sp19/projects/proj4/}}} \cite{cs61c}

\section{Διάρθρωση της Διπλωματικής Εργασίας}

