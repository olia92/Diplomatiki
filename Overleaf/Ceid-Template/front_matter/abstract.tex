\begin{abstract}
Στις μέρες μας, τα Συνελικτικά Νευρωνικά Δίκτυα είναι δημοφιλή για την ταξινόμηση και την αναγνώριση εικόνων. Προτιμούμε να τα χρησιμοποιούμε επειδή επιτυγχάνουν υψηλή ακρίβεια αξιοποιώντας τις εγγενείς ιδιότητες των εικόνων. Ένα σημαντικό μειονέκτημα των CNN είναι ότι εκτελούν πολλούς και πολύπλοκους υπολογισμούς που κοστίζουν πολύ χρόνο, ενέργεια και πόρους. Η καλύτερη λύση που μπορούμε να προτείνουμε είναι να εκμεταλλευτούμε τις ιδιότητες των Field Programmable Gate Arrays. Οι FPGA είναι εξειδικευμένες στην επιτάχυνση των υπολογισμών και καταναλώνουν λιγότερη ενέργεια από τις μονάδες επεξεργασίας γραφικών ή τις κεντρικές μονάδες επεξεργασίας. 
Παρουσιάζουμε ένα πλαίσιο γραμμένο σε C++ που μπορεί να υιοθετήσει πυρήνες FPGA για την επιτάχυνση υπολογισμών όπως οι πολλαπλασιασμοί πινάκων. Συνδέσαμε μια διαθέσιμη υλοποίηση πολλαπλασιασμού matrix με πρόσθεση με το πλαίσιο και το δοκιμάσαμε σε μια πλατφόρμα Trenz. Εκτός από αυτό, υλοποιήσαμε ένα γρήγορο και cache-aware πλαίσιο εφαρμόζοντας OpenMP, σημαίες επιλογής του GCC, μεταβλητές περιβάλλοντος OpenMP και χαρακτηριστικά από την C++17. Το πλαίσιο CNN μας δοκιμάστηκε σε μια αρχιτεκτονική LeNet-5 χρησιμοποιώντας το σύνολο δεδομένων MNIST που περιέχει L1, L2 Regularizations, Vanilla, Momentum, Momentum with Nesterov Updates, He-et-al weight initialization, Fisher-Yates shuf- fle, Stochastic Gradient Descent techniques που όλα υλοποιήθηκαν από το μηδέν. Επιπλέον, υλοποιήσαμε 3 τρόπους φόρτωσης του συνόλου δεδομένων MNIST, καθώς και, naive, cache blocking, OpenMP και Hybrid cache blocking with OpenMP σε αλγορίθμους πολλαπλασιασμού πινάκων, transpose και copy, που δοκιμάσαμε και διερευνήσαμε τη συμπεριφορά τους μεταξύ των μεγεθών των μίνι-πακέτων και του αριθμού των χρησιμοποιούμενων νημάτων. 
Εκτός από όλα τα προαναφερθέντα που φτιάχτηκαν από το μηδέν, χρησιμοποιήσαμε το Xilinx Vivado SDK για να φτιάξουμε ένα bare-metal C++ project με το κατάλληλο σενάριο linker cache size και προσαρμόσαμε τον πολλαπλασιασμό μήτρας με πρόσθεση κώδικα στο πλαίσιο μας. Στη συνέχεια, προγραμματίσαμε την πλατφόρμα Trenz που περιέχει μια CPU ARM και έναν επιταχυντή FPGA. Ως αποτέλεσμα, πετύχαμε 4,3x-8,5x καλύτερη επίδοση χρησιμοποιώντας μια FPGA για την επιτάχυνση του πολλαπλασιασμού πινάκων με πρόσθεση από ό,τι χρησιμοποιώντας μια αφελής ή μια κρυφή μνήμη αποκλεισμού υλοποιήσεις ενός νήματος και σε συγκεκριμένες αδικαιολόγητες(έλλειψη multi-threading στο Trenz) περιπτώσεις, ανάλογα με το μέγεθος της μίνι παρτίδας πολλαπλών αλγορίθμων OpenMP(έως 1,27x) ή υβριδικών αλγορίθμων(έως 2,27x) σε μια CPU.

   \begin{keywords}
   Παράλληλος Προγραμματισμός, Συνελικτικά Νευρωνικά Δίκτυα, \tl{OpenACC}, \tl{GPU}
   \end{keywords}
\end{abstract}



\begin{abstracteng}
\tl{Nowadays, Cellular Neural Networks are popular for image classification and recognition. We prefer to use them because they achieve high accuracy by exploiting the intrinsic properties of images. A major drawback of CNNs is that they perform many complex computations that cost a lot of time, energy and resources. The best solution we can propose is to exploit the properties of Field Programmable Gate Arrays. FPGAs are specialized in accelerating computations and consume less power than graphics processing units or central processing units. 
We present a framework written in C++ that can adopt FPGA cores to accelerate computations such as matrix multiplications. We link an available matrix multiplication implementation with addition to the framework and test it on a Trenz platform. In addition to this, we implemented a fast and cache-aware framework by implementing OpenMP, GCC selection flags, OpenMP environment variables and features from C++17. Our CNN framework was tested on a LeNet-5 architecture using the MNIST dataset containing L1, L2 Regularizations, Vanilla, Momentum, Momentum with Nesterov Updates, He-et-al weight initialization, Fisher-Yates shuf- fle, Stochastic Gradient Descent techniques all implemented from scratch. In addition, we implemented 3 ways of loading the MNIST dataset, as well as, naive, cache blocking, OpenMP and Hybrid cache blocking with OpenMP in matrix multiplication algorithms, transpose and copy, which we tested and investigated their behavior between mini-packet sizes and the number of threads used. 
In addition to all the aforementioned built from scratch, we used the Xilinx Vivado SDK to build a bare-metal C++ project with the appropriate cache size linker script and adapted matrix multiplication by adding code to our framework. We then programmed the Trenz platform containing an ARM CPU and an FPGA accelerator. As a result, we achieved 4.3x-8.5x better performance using an FPGA to accelerate matrix multiplication with addition than using a naive or single-threaded cache blocking implementations and in specific unjustified(lack of multi-threading in Trenz) cases, depending on the size of the mini-batch of OpenMP multi-algorithm(up to 1.27x) or hybrid algorithms(up to 2.27x) on a CPU.}

   \begin{keywordseng}
    \tl{Parallel Programming, Convolutional Neural Networks, OpenACC, GPU}
   \end{keywordseng}

\end{abstracteng}