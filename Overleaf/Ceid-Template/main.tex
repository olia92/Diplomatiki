%%
\documentclass[11pt,a4paper,english,greek,twoside]{ceid-thesis}

\usepackage{xurl}
\usepackage{graphicx}
\usepackage{epstopdf}
\usepackage{indentfirst}
\usepackage{verbatim}
\usepackage{amsmath}
\usepackage{amsthm}
\usepackage{amssymb}
\usepackage{latexsym}
\bibliographystyle{static/hellas}
\usepackage{hyphenat}
\usepackage{makeidx}
\addto\captionsgreek{%
  \renewcommand{\indexname}{Ευρετήριο όρων}%
}
\makeindex

% 1.5 spacing
\renewcommand{\baselinestretch}{1.2}

% latin text (and greek text)
\newcommand{\tl}[1]{\textlatin{#1}}
\newcommand{\tg}[1]{\textgreek{#1}}

% typeset short english phrases
\newcommand{\en}[1]{\foreignlanguage{english}{#1}}

% typeset source code
\newcommand{\src}[1]{{\tt\en{#1}}}

% typeset a backslash
\newcommand{\bkslash}{\en{\symbol{92}}}

\newtheorem{definition}{Ορισμός}
\newtheorem{proposition}{Πρόταση}
\newtheorem{theorem}{Θεώρημα}
\newtheorem{corollary}{Συμπέρασμα}
\newtheorem{lemma}{Λήμμα}
\newtheorem{example}{Παράδειγμα}
\newtheorem{remark}{Σημείωση}
\newtheorem{notation}{Συμβολισμός}
\newtheorem{law}{Νόμος}
\renewcommand{\thedefinition}{\arabic{chapter}.\arabic{definition}}
\renewcommand{\theproposition}{\arabic{chapter}.\arabic{proposition}}
\renewcommand{\thetheorem}{\arabic{chapter}.\arabic{theorem}}
\renewcommand{\thecorollary}{\arabic{chapter}.\arabic{corollary}}
\renewcommand{\thelemma}{\arabic{chapter}.\arabic{lemma}}
\renewcommand{\theexample}{\arabic{chapter}.\arabic{example}}
\newcommand{\set}[1]{\left\{#1\right\}}
\newcommand{\To}{\Longrightarrow}
\newcommand{\xml}{\en{XML}}

\selectlanguage{greek}

\hyphenation{ο-ποί-α}

%%%%%%%%%%%%%%%%%%%%%%%%%%%%%%%%%%%%%%%%%%%%%%%%%%%%%
%% THESIS INFO 
%%
%
% Τίτλος Πτυχιακής Εργασίας
	\title{Αναζήτηση και Εξόρυξη Πληροφορίας σε Μεγάλες Βάσεις Αδόμητων Δεδομένων με Μεθόδους Παράλληλης Επεξεργασίας}
% "του" ή "της", ανάλογα με το φύλο του σπουδαστή
	\edef\toutis{της}
% Ονοματεπώνυμο σπουδαστή (ΚΕΦΑΛΑΙΑ, γενική πτώση)
	\edef\authorNameCapital{ΟΛΥΜΠΙΑΣ ΤΣΑΜΟΥ}
% Ονοματεπώνυμο σπουδαστή (πεζά, ονομαστική πτώση)
	\author{Ολυμπία Τσάμου}
% Ονοματεπώνυμο Επιβλέποντα Καθηγητή
	\supervisor{Ευάγγελος Δερματάς}
    \edef\supervisorTitle{Αναπληρωτής Καθηγητής}
% Ονοματεπώνυμο Επιβλέποντα Καθηγητή
	\supervisorSecond{Θεόδωρος Αντωνακόπουλος}
    \edef\supervisorSecondTitle{Καθηγητής}
% "Επιβλέπων" ή "Επιβλέπουσα", ανάλογα με το φύλο του Επιβλέποντα Καθηγητή
	\edef\supervisorMaleFemale{Επιβλέπων}
% Τόπος, μήνας και έτος
	\edef\thesisPlaceDate{Πάτρα, Δεκέμβριος 2022}
% Ημερομηνία Εξέτασης
	\edef\examinationDate{........./........./.........}
% Έτος Copyright
	\edef\copyrightYear{2022}
% Ονοματεπώνυμο 1ου εξεταστή
	\epitropiF{Θεόδωρος Αντωνακόπουλος}
% Τίτλος 1ου εξεταστή
	\edef\epitropiFTitle{Καθηγητής}
% Ονοματεπώνυμο 2ου εξεταστή
	\epitropiS{****}
% τίτλος 2ου εξεταστή
	\edef\epitropiSTitle{****}
%%%%%%%%%%%%%%%%%%%%%%%%%%%%%%%%%%%%%%%%%%%%%%%%%%%%%

% \usepackage[unicode]{hyperref}
\begin{document}
\selectlanguage{greek}
\maketitle

\frontmatter
% Περίληψη
	\begin{abstract}
Στις μέρες μας, τα Συνελικτικά Νευρωνικά Δίκτυα είναι δημοφιλή για την ταξινόμηση και την αναγνώριση εικόνων. Προτιμούμε να τα χρησιμοποιούμε επειδή επιτυγχάνουν υψηλή ακρίβεια αξιοποιώντας τις εγγενείς ιδιότητες των εικόνων. Ένα σημαντικό μειονέκτημα των CNN είναι ότι εκτελούν πολλούς και πολύπλοκους υπολογισμούς που κοστίζουν πολύ χρόνο, ενέργεια και πόρους. Η καλύτερη λύση που μπορούμε να προτείνουμε είναι να εκμεταλλευτούμε τις ιδιότητες των Field Programmable Gate Arrays. Οι FPGA είναι εξειδικευμένες στην επιτάχυνση των υπολογισμών και καταναλώνουν λιγότερη ενέργεια από τις μονάδες επεξεργασίας γραφικών ή τις κεντρικές μονάδες επεξεργασίας. 
Παρουσιάζουμε ένα πλαίσιο γραμμένο σε C++ που μπορεί να υιοθετήσει πυρήνες FPGA για την επιτάχυνση υπολογισμών όπως οι πολλαπλασιασμοί πινάκων. Συνδέσαμε μια διαθέσιμη υλοποίηση πολλαπλασιασμού matrix με πρόσθεση με το πλαίσιο και το δοκιμάσαμε σε μια πλατφόρμα Trenz. Εκτός από αυτό, υλοποιήσαμε ένα γρήγορο και cache-aware πλαίσιο εφαρμόζοντας OpenMP, σημαίες επιλογής του GCC, μεταβλητές περιβάλλοντος OpenMP και χαρακτηριστικά από την C++17. Το πλαίσιο CNN μας δοκιμάστηκε σε μια αρχιτεκτονική LeNet-5 χρησιμοποιώντας το σύνολο δεδομένων MNIST που περιέχει L1, L2 Regularizations, Vanilla, Momentum, Momentum with Nesterov Updates, He-et-al weight initialization, Fisher-Yates shuf- fle, Stochastic Gradient Descent techniques που όλα υλοποιήθηκαν από το μηδέν. Επιπλέον, υλοποιήσαμε 3 τρόπους φόρτωσης του συνόλου δεδομένων MNIST, καθώς και, naive, cache blocking, OpenMP και Hybrid cache blocking with OpenMP σε αλγορίθμους πολλαπλασιασμού πινάκων, transpose και copy, που δοκιμάσαμε και διερευνήσαμε τη συμπεριφορά τους μεταξύ των μεγεθών των μίνι-πακέτων και του αριθμού των χρησιμοποιούμενων νημάτων. 
Εκτός από όλα τα προαναφερθέντα που φτιάχτηκαν από το μηδέν, χρησιμοποιήσαμε το Xilinx Vivado SDK για να φτιάξουμε ένα bare-metal C++ project με το κατάλληλο σενάριο linker cache size και προσαρμόσαμε τον πολλαπλασιασμό μήτρας με πρόσθεση κώδικα στο πλαίσιο μας. Στη συνέχεια, προγραμματίσαμε την πλατφόρμα Trenz που περιέχει μια CPU ARM και έναν επιταχυντή FPGA. Ως αποτέλεσμα, πετύχαμε 4,3x-8,5x καλύτερη επίδοση χρησιμοποιώντας μια FPGA για την επιτάχυνση του πολλαπλασιασμού πινάκων με πρόσθεση από ό,τι χρησιμοποιώντας μια αφελής ή μια κρυφή μνήμη αποκλεισμού υλοποιήσεις ενός νήματος και σε συγκεκριμένες αδικαιολόγητες(έλλειψη multi-threading στο Trenz) περιπτώσεις, ανάλογα με το μέγεθος της μίνι παρτίδας πολλαπλών αλγορίθμων OpenMP(έως 1,27x) ή υβριδικών αλγορίθμων(έως 2,27x) σε μια CPU.

   \begin{keywords}
   Παράλληλος Προγραμματισμός, Συνελικτικά Νευρωνικά Δίκτυα, \tl{OpenACC}, \tl{GPU}
   \end{keywords}
\end{abstract}



\begin{abstracteng}
\tl{Nowadays, Cellular Neural Networks are popular for image classification and recognition. We prefer to use them because they achieve high accuracy by exploiting the intrinsic properties of images. A major drawback of CNNs is that they perform many complex computations that cost a lot of time, energy and resources. The best solution we can propose is to exploit the properties of Field Programmable Gate Arrays. FPGAs are specialized in accelerating computations and consume less power than graphics processing units or central processing units. 
We present a framework written in C++ that can adopt FPGA cores to accelerate computations such as matrix multiplications. We link an available matrix multiplication implementation with addition to the framework and test it on a Trenz platform. In addition to this, we implemented a fast and cache-aware framework by implementing OpenMP, GCC selection flags, OpenMP environment variables and features from C++17. Our CNN framework was tested on a LeNet-5 architecture using the MNIST dataset containing L1, L2 Regularizations, Vanilla, Momentum, Momentum with Nesterov Updates, He-et-al weight initialization, Fisher-Yates shuf- fle, Stochastic Gradient Descent techniques all implemented from scratch. In addition, we implemented 3 ways of loading the MNIST dataset, as well as, naive, cache blocking, OpenMP and Hybrid cache blocking with OpenMP in matrix multiplication algorithms, transpose and copy, which we tested and investigated their behavior between mini-packet sizes and the number of threads used. 
In addition to all the aforementioned built from scratch, we used the Xilinx Vivado SDK to build a bare-metal C++ project with the appropriate cache size linker script and adapted matrix multiplication by adding code to our framework. We then programmed the Trenz platform containing an ARM CPU and an FPGA accelerator. As a result, we achieved 4.3x-8.5x better performance using an FPGA to accelerate matrix multiplication with addition than using a naive or single-threaded cache blocking implementations and in specific unjustified(lack of multi-threading in Trenz) cases, depending on the size of the mini-batch of OpenMP multi-algorithm(up to 1.27x) or hybrid algorithms(up to 2.27x) on a CPU.}

   \begin{keywordseng}
    \tl{Parallel Programming, Convolutional Neural Networks, OpenACC, GPU}
   \end{keywordseng}

\end{abstracteng}
% Αφιέρωση
	\thesisDedication{στους γονείς μου}
% Ευχαριστίες
	\include{front_matter/acknowledgements}
% Πίνακας Περιεχομένων
	\tableofcontents
% Κατάλογος Σχημάτων
	\listoffigures
% Κατάλογος Πινάκων
	\listoftables

%%%%%%%%%%%%%%%%%%%%%%%%%%%%%%%%%%%%%%%%%%%%%%%%%%%%%
%% INCLUDE YOUR CHAPTERS/SECTIONS HERE
%%
\mainmatter
% Εισαγωγή
	\chapter{Εισαγωγή}

\section{Ορισμός του Προβλήματος}
Εφαρμογή τεχνικών βελτιστοποίησης σε πρόβλημα ταξινόμησης εικόνων κάνοντας χρήση
ενός Συνελικτικού Νευρωνικού Δικτύου

\section{Στόχοι της Διπλωματικής Εργασίας}

Γίνεται έρευνα σχετικά με τις διαφορετικές αρχιτεκτονικές νευρωνικών δικτύων και τις μεθόδους εκπαίδευσης, ωστόσο, μια άλλη κρίσιμη πτυχή των νευρωνικών δικτύων είναι, δεδομένου ενός εκπαιδευμένου δικτύου, η γρήγορη και ακριβής ταξινόμηση εικόνων. Σε αυτό το \en{project}, δίνετε ένα εκπαιδευμένο νευρωνικό δίκτυο που ταξινομεί εικόνες \(32\times32\) \en{RGB} σε 10 κατηγορίες. Οι εικόνες ανήκουν στο \en{dataset CIFAR-10 \cite{cifar-10}}. Μας δίνετε ο αλγόριθμος εμπρόσθιας διάδοσης του νευρωνικού δικτύου και τα τελικά βάρη του δικτύου.\cite{cs61c} Σκοπός είναι η βελτίωση της ταχύτητας της εμπρόσθιας διάδοσης ώστε να γίνει ταξινόμηση με πιο γρήγορο ρυθμό. 

Βασίστικε πάνω στο \en{project} του \en{Berkeley {\url{https://inst.eecs.berkeley.edu/~cs61c/sp19/projects/proj4/}}} \cite{cs61c}

\section{Διάρθρωση της Διπλωματικής Εργασίας}


% Κεφάλαια
	\chapter{Θεωρητικό υπόβαθρο}
Κατηγοριοποίηση Εικόνων με χρήση Συνελικτικών Νευρωνικών Δικτύων  

\section{Αναγνώριση Εικόνων}
\subsubsection{Πώς ένας υπολογιστής αναγνωρίζει εικόνες\;}
Η ταξινόμηση εικόνων περιγράφει ένα πρόβλημα στο οποίο σε ένα υπολογιστή δίνεται μία εικόνα και πρέπει να καταλάβει τι απεικονίζει (από ένα σύνολο πιθανών κατηγοριών)
Σήμερα, τα Συνελικτικά Νευρωνικά Δίκτυα \en{(CNNs)} αποτελούν  μια πολύ καλή  προσέγγιση αυτού το προβλήματος. Γενικά, τα νευρωνικά δίκτυα υποθέτουν πως υπάρχει κάποια συνάρτηση από την είσοδο (π.χ. εικόνες) σε μία έξοδο (π.χ. ένα σύνολο κατηγοριών εικόνων). Ενώ οι κλασσικοί αλγόριθμοι προσπαθούν να κωδικοποιήσουν  κάποια πληροφορία του πραγματικού κόσμου στη συνάρτηση τους, τα \en{CNN} μαθαίνουν την συνάρτηση δυναμικά  από ένα σύνολο ταξινομημένων εικόνων \en{(labelled images)}—αυτή η διαδικασία ονομάζετε εκπαίδευση. Μόλις καταλήξει σε μια σταθερή συνάρτηση (δηλαδή σε μια προσέγγιση αυτής), μπορεί να εφαρμόσει τη συνάρτηση σε εικόνες που δεν έχει ξαναδεί.

\subsubsection{Τι μπορεί να κάνει ένα νευρωνικό δίκτυο;}
Ένα νευρωνικό δίκτυο αποτελείται από πολλαπλά επίπεδα.  Κάθε επίπεδο λαμβάνει έναν πολυδιάστατο πίνακα αριθμών ως είσοδο και παράγει έναν άλλο πολυδιάστατο πίνακα αριθμών ως έξοδο (ο οποίος στη συνέχεια γίνεται η είσοδος του επόμενου επιπέδου). Κατά την ταξινόμηση εικόνων, η είσοδος του πρώτου επιπέδου είναι η εικόνα εισόδου  (π.χ.\( 32\times32\times3\) αριθμοί για εικόνες \(32\times32\) pixel με 3 κανάλια χρώματος), ενώ η έξοδος του τελευταίου επιπέδου αποτελείται  ένα σύνολο πιθανοτήτων των διαφόρων κατηγοριών (π.χ., \(1\times1\times10\) αριθμοί αν υπάρχουν \(10\) κατηγορίες).
 
Κάθε επίπεδο έχει ένα σύνολο από βάρη που σχετίζονται με αυτό — αυτά τα βάρη είναι που “μαθαίνει” το νευρωνικό όταν του δοθούν δεδομένα εκπαίδευσης. Ανάλογα με το επίπεδο, τα βάρη έχουν διαφορετικές ερμηνείες, αλλά δεν είναι αντικείμενο μελέτης του συγκεκριμένου \en{project}, φτάνει να γνωρίζουμε ότι κάθε επίπεδο λαμβάνει μία είσοδο, εκτελεί κάποια διεργασία σε αυτή, που εξαρτάται από τα βάρη και παράγει μια έξοδο. Αυτό το βήμα ονομάζεται εμπρόσθια διάδοση: παίρνουμε μία είσοδο και την προωθούμε στο δίκτυο, παράγοντας το επιθυμητό αποτέλεσμα ως έξοδο. Η εμπρόσθια διάδοση είναι το μόνο που χρειάζεται για την ταξινόμηση εικόνων σε ένα ήδη εκπαιδευμένο \en{CNN}.
 
Στην πράξη, ένα νευρωνικό δίκτυο αποτελεί μια πολύ απλή μηχανή αναγνώρισης προτύπων (με εξαιρετικά περιορισμένη χωρητικότητα), αλλά μπορεί να είναι αρκετά παράξενο αυτό που καταλήγει να αναγνωρίσει. Για παράδειγμα, κάποιος μπορεί να εκπαιδεύσει ένα νευρωνικό δίκτυο να αναγνωρίζει τη διαφορά μεταξύ “σκύλων” και “λύκων”, και να δουλέψει καλά κοιτώντας το χιόνι και το δάσος στο φόντο των φωτογραφιών με τους λύκους.

\section{Συνελικτικά Νευρωνικά Δίκτυα}
Συνελικτικά Νευρωνικά Δίκτυα \index{Νευρωνικά Δίκτυα}, \tl{Stanford} \cite{cs231n}.

Πηγή:
\begin{tabbing}
\src{https://cs231n.github.io/neural-networks-1/ }
\end{tabbing}

\subsection{Τι είναι τα Νευρωνικά Δίκτυα}

\subsubsection{Μοντελοποίηση ενός νευρώνα}

Οι νευρώνες είναι εμπνευσμένοι από τους βιολογικούς νευρώνες του ανθρώπινου νευρικού συστήματος. Παρακάτω φαίνεται μια απεικόνιση ενός βιολογικού νευρώνα και η μαθηματική μοντελοποίησή του. 

\begin{figure}[!ht] 
\centering
\includegraphics[width=\textwidth]{static/figures/cs231n_neuron.png} 
\caption{\en{A cartoon drawing of a biological neuron (left) and its mathematical model (right).}}
\label{neuron model}
\end{figure}

Κάθε νευρώνας λαμβάνει σήματα από τους δενδρίτες και παράγει ένα σήμα εξόδου στον άξονα. Στη συνέχεια ο άξονας διακλαδίζεται μέσω συνάψεων σε δενδρίτες άλλων νευρώνων. Στο υπολογιστικό μοντέλο τα σήματα ($x_0$) στον άξονα, αλληλοεπιδρούν πολλαπλασιαστικά μέσω των συνάψεων με τους δενδρίτες ($w_0 x_0$) . Οι συνάψεις θεωρούμε πως είναι τα εκπαιδεύσιμα  στοιχεία (βάρη $w$)  τα οποία ελέγχουν την επιρροή του ενός νευρώνα σε κάποιον άλλο. Στο κυρίως μέρος του νευρώνα, αθροίζονται τα σήματα από τους δενδρίτες. Εάν αυτό το άθροισμα ξεπερνά ένα συγκεκριμένο κατώφλι, ο νευρώνας ενεργοποιείται και στέλνει σήμα στον άξονα εξόδου. Στο μαθηματικό μοντέλο ο ακριβής χρόνος που παράγονται τα σήματα δεν έχει σημασία, η συχνότητα ενεργοποίησης του νευρώνα μας ενδιαφέρει. Ο τρόπος με τον οποίο το αναπαριστούμε  αυτό είναι με μια συνάρτηση ενεργοποίησης. Η πιο συνηθισμένη συνάρτηση είναι η σιγμοειδής, η οποία έχει σαν είσοδο μια πραγματική τιμή (την ισχύ του σήματος μετά το άθροισμα) και το περιορίζει στο διάστημα μεταξύ 0 και 1. Με άλλα λόγια, κάθε νευρώνας εκτελεί το εσωτερικό γινόμενο της εισόδου με τα βάρη προσθέτοντας και μία σταθερά (\en{bias}) και εφαρμόζει μία μη γραμμική συνάρτηση, σε αυτή την περίπτωση η σιγμοειδής  $\sigma(x)=1/(1+e-x)$ .


	\include{body_matter/chap3}
	\include{body_matter/chap4}
	\include{body_matter/chap5}
	\include{body_matter/chap6}
	\include{body_matter/chap7}
	\include{body_matter/chap8}
	\include{body_matter/chap9}
% % Παραρτήματα
% 	\appendix
% 	\include{back_matter/appA}
% 	\include{back_matter/appB}	
% 	\cleardoublepage
% Βιβλιογραφία - Αναφορές
    \nocite{*}%τυπώνει όλες τις αναφορές
	\bibliography{back_matter/references}
% % Συντομογραφίες
% 	\newcommand{\abbrevEN}[2]{\en{#1} \> \en{#2}\\ }
\newcommand{\abbrevGR}[2]{#1 \> #2\\ }

\chapter*{Συντομογραφίες}

\begin{tabbing}
%ta 'a' rythmizoun to platos ton dyo stilon
  aaaaaaaaaaaaaaaaa \= aaaaaaaaaaaaaaaaaaaaaa\kill
  \abbrevGR{βλπ}{βλέπε}
  \abbrevEN{CNN}{Convolutional Neural Network}
\end{tabbing}
% % Γλωσσάριο
% 	\newcommand{\gloss}[2]{#1 \> \en{#2}\\ }

\chapter*{Απόδοση ξενόγλωσσων όρων}

\begin{tabbing}
%ta 'a' rythmizoun to platos ton dyo stilon
  aaaaaaaaaaaaaaaaaaaaaaaaaaaaaaaaaaa \= aaaa\kill
  \Large\textbf{Απόδοση} \> \Large\textbf{Ξενόγλωσσος όρος} \\
  \gloss{εμπρόσθια διάδοση}{forward pass}
  
\end{tabbing}
%%%%%%%%%%%%%%%%%%%%%%%%%%%%%%%%%%%%%%%%%%%%%%%%%%%%
% \backmatter
% % Ευρετήριο Όρων
% 	\printindex
% % 	\cleardoublepage

	\pagebreak
	\thispagestyle{empty}
\end{document}