\chapter{Υλοποίηση \en{CNN}}

\section{Κατανόηση του κώδικα}

Ο κώδικας που χρησιμοποιήσαμε προέρχεται από \en{project} του πανεπιστημίου του \en{Berkeley} στο \en{github} \cite{sp19-proj4}. Περιλαμβάνει τα εξής αρχεία:
\selectlanguage{english}
\begin{lstlisting}[backgroundcolor=\color{backcolour},breakatwhitespace=false]
    
    Makefile
    layers.h
    snapshot/
    layers_baseline.c
    test/
    benchmark.c
    network.c
    volume.c
    network.h
    volume.h
    huge_test.sh
    network_baseline.c
    layers.c
    run_test.sh
    
\end{lstlisting}
\selectlanguage{greek}

Tα αρχεία στα οποία έγιναν μετατροπές για παραλληλοποίηση του κώδικα με χρήση της \en{OpenACC} είναι τα :\emph{ \en{layers.c, network.c, volume.c}}. Tα αρχεία με όνομα \en{\emph{*\_baseline.c}} αποτελούν αρχεία αναφοράς ώστε να μετρήσουμε την επιτάχυνση του κώδικα.

\subsection{Επισκόπηση κώδικα}
\
Στον κώδικα χρησιμοποιούμε τύπους δεδομένων που είναι οργανωμένοι σε \en{structs}.
Πρώτος τύπος δεδομένων είναι το \textbf{\en{volume\_t}} το οποίο αναπαριστά τρισδιάστατους πίνακες (ή Όγκους) δεδομένα τύπου \textbf{\en{double}}. 
\empty
\selectlanguage{english}

\hrule

\section{Table}

    \begin{table}[ht]
    \centering
    \begin{tabular}{l | l | l}
    A & B & C \\
    \hline
    1 & 2 & 3 \\
    4 & 5 & 6
    \end{tabular}
    \caption{very basic table}
    \label{tab:abc}
\end{table}
\hrule
\begin{table}[ht]
    \begin{subtable}[h]{0.45\textwidth}
        \centering
        \begin{tabular}{l | l | l}
        Day & Max Temp & Min Temp \\
        \hline \hline
        Mon & 20 & 13\\
        Tue & 22 & 14\\
        Wed & 23 & 12\\
        Thurs & 25 & 13\\
        Fri & 18 & 7\\
        Sat & 15 & 13\\
        Sun & 20 & 13
        \end{tabular}
        \caption{First Week}
        \label{tab:week1}
    \end{subtable}
    \hfill
    \begin{subtable}[h]{0.45\textwidth}
        \centering
        \begin{tabular}{l | l | l}
        Day & Max Temp & Min Temp \\
        \hline \hline
        Mon & 17 & 11\\
        Tue & 16 & 10\\
        Wed & 14 & 8\\
        Thurs & 12 & 5\\
        Fri & 15 & 7\\
        Sat & 16 & 12\\
        Sun & 15 & 9
        \end{tabular}
        \caption{Second Week}
        \label{tab:week2}
    \end{subtable}
    \caption{Max and min temps recorded in the first two weeks of July}
    \label{tab:temps}
\end{table}

\hrule


\section{Cite}

Cite see page \parencite[see][p10]{latexcompanion}.

Cite Compare \parencite[compare][]{knuthwebsite}.


Cite page \parencite[e.g.][page 300]{einstein}.
\selectlanguage{greek}